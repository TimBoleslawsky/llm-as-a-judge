\section{Problem Definition: Evaluating the Request to LLM to Response Workflow}

We formalize the problem to unify different evaluation goals as well as different evaluation paradigms, and to support principled meta-evaluation of evaluators (``judges'').

\subsection{Formal Setting}
Let:
\begin{itemize}
	\item Request (prompt) space: $\mathcal{X}$ (task inputs~/~instructions)~\cite{srivastava2022bigbench, liang2022helm}.
	\item Response (candidate output) space: $\mathcal{Y}$.
	\item Optional auxiliary context (tools, retrieved docs, conversation history): $\mathcal{C}$~\cite{liang2022helm}.
	\item A (possibly stochastic) generative LLM\@: $G_{\theta}: \mathcal{X} \times \mathcal{C} \to \mathcal{Y}$.
	\item We assume a latent human quality function (unobserved): $q^{*}: \mathcal{X} \times \mathcal{Y} \to \mathbb{R}$ (scalar utility encapsulating correctness, relevance, style, safety)~\cite{stiennon2020learning, ouyang2022training}. Why accounting for the latent human quality function $q^{*}$ is important will become clear in the section on evaluator construction.
	\item Ground-truth task signal (when available): for example structured labels. For programming tasks, objective execution-based signals (unit test pass rate) ground $g(x,y)$ in functional correctness~\cite{chen2021evaluating}.
\end{itemize}

Instead of defining the evaluation problem as a being solvable by a single judge
\begin{equation}
	J: (\mathcal{X},\mathcal{C},\mathcal{Y}) \to \mathbb{R},
\end{equation}  
We factor the evaluation problem into three subproblems that isolate distinct failure modes:

\begin{itemize}
    \item Is the prompt (X)/context (C) itself adequate and usable? (Not a model output issue.)
    \item Is the produced response (Y) high quality in isolation? (Fluency / style / correctness independent of instruction grounding.)
    \item Does (Y) faithfully fulfill the instruction given (X) and make proper use of (or remain consistent with) (C)? (Mapping fidelity / alignment.)
\end{itemize}
This mirrors multi-axis scenario evaluation (HELM) and systematic separation of capability, alignment, robustness, and fairness dimensions.~\cite{liang2022helm, srivastava2022bigbench}.

\subsection{Evaluation Paradigms}
Given the different ways to evaluate artefacts outlined in the literature, we can categorize evaluation paradigms as follows:
\paragraph{Pointwise Scoring.} Produce a score $s = J(x,y)$ and compare to a human or rubric-derived reference signal (scalar or categorical)~\cite{fu2023gptscore, li2024_llmsasjudges}. 
\paragraph{Pairwise Preference.} Given $(x, y_i, y_j)$, provide a preference for either $y_i$ or $y_j$~\cite{zheng2023judgelm}. This approach is extended by one, \emph{likwise / tournament aggregation} methods. For a set $\{y_1,\dots,y_k\}$, produce a permutation or aggregate scores using Elo~\cite{elo1978rating, zheng2023judgelm} or Bradley-Terry / Plackett-Luce style models~\cite{bradley1952rank} to obtain stable rankings, as operationalized in live arenas~\cite{zheng2023judgelm}. And two, \emph{reference-guided grading}. Given a reference response $y^{*}$ (e.g., ground-truth solution), provide preference for candidates $y_i$ and $y_j$ via similarity or entailment metrics~\cite{zheng2023judgelm}. Notably, pairwise preferences can be represented probabilistically as $p = J(x,y_i,y_j) \in [0,1]$ indicating the likelihood that humans would prefer $y_i$ over $y_j$, enabling uncertainty quantification and calibration assessment~\cite{stiennon2020learning, ouyang2022training}.

